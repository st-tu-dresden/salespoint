\section{Quantity and Money}
\label{sec:quantity}
\code{Quantity} is used to represent amounts of anything.
Three attributes allow \code{Quantity} to specify everything: a numerical value (\code{BigDecimal}), a (measurement) unit or metric (\code{Metric}), and a type specifying the rounding of the numerical type (\code{RoundingStrategy}), as can be seen in Figure \ref{quantity_overview}.

\code{Quantity} objects are immutable and the class implements the \code{Comparable} interface.

\begin{figure}[ht]
	\centering
  \includegraphics[width=1.0\textwidth]{images/Quantity_Overview.eps}
	\label{quantity_overview}
	\caption{Quantity - Class Overview}
\end{figure}

\subsection{\code{BigDecimal} - Representing numerical values}
\code{BigDecimal} was chosen as datatype for the \code{amount} attribute (see Figure~\ref{quantity_overview}) over \code{float} or \code{double} because of its arbitraty precision.
Moreover, objects of \code{BigDecimal} are immutable and the \code{BigDecimal} class provides operations for including, but not limited to: arithmetic, rounding, and comparison.

\subsection{\code{Metric} - What is represented}
The composite type \code{Metric} contains all information pertaining to the unit or metric of the represented object.
Examples for units or metrics are: m (meter), s (second), pcs (pieces).
For example consider the unit of length "meter": represented by an object of the class \code{Metric} the symbol would be set to "\code{m}" and the name to "\code{meter}".
Furthermore, an object of type \code{Metric} has a description field, to explain the meaning of the metric in detail.
For the example of a meter a possible description could be "\code{The meter is the length of the path travelled by light in vacuum during a time interval of 1/299 792 458 of a second.}".

Convenience instances exist for euros, pieces and units, namely \code{EURO}, \code{PIECES}, and \code{UNIT} (see Figure \ref{quantity_overview}).

\subsection{\code{RoundingStrategy} - How to handle half a person}
When handling quantities of unkown metric, standard rounding rules cannot always be employed.
The case of natural persons is just one example, when rounding rules have to be restricted to yield a useful result.
You can round in four general directions: away from zero (\code{RoundUpStrategy}), towards zero (\code{RoundDownStrategy}), towards positive infinity (\code{RoundCeilStrategy}), and towards negative infinity (\code{RoundFloorStrategy}).

Additionally, you can specify the digits after the decimal delimiter (\code{numberOfDigits} in Figure~\ref{quantity_overview}).
Monetary values in \euro{} or \$US are often just represented with two digits after the decimal delimiter.
Other values, such as kilo grams may be required to be specified to four digits after the decimal delimiter or even more.
In case of (natural) persons, the digits after the decimal delimiter is usually zero, except you are working in statistics (1.45 children per couple) or you are a serial killer dismembering your victims.\footnote{Of course you can only kill an integral number of people. However, only a part of a victims body may be found. Anyway, in case you have not noticed: that was a joke. Haha. Fat chance.}

The third parameter for rounding is the rounding digit, i.e. the number specifying when you round up or down.
Usually, this number is five, but it can be specified using the attribute \code{roundingDigit} and the \code{RoundStrategy} as strategy.
In case of persons, it is one: if you have $n.0\,persons$, you round down, otherwise up.
If you are calculating a capacity for persons, you will have to round down.
This can be achieved by specifying the correct rounding direction.
\\

Sometimes, it is necessary to round a number to a nearest ``step'', i.e. if you sell something in packs of $50$, and someone punches in $40$, you will have to round up to $50$.
So your rounding step is $50$.
Another example is material, which is sold by the meter or yard.
You have to round the amount specified by your customer accordingly.
Of course, a rounding step can be smaller than $1$, i.e. $0.25$.
For this purpose, \code{RoundDownStepStrategy} and \code{RoundUpStepStrategy} exist, depending on wether rounding to the nearest step smaller or bigger than the current value is intended.
The step is set using the attribute \code{roundingStep}.
\\

Two convenience rounding strategies exist so far: \code{RoundingStrategy.MONETARY} rounding with four digits after the decimal delimiter and rounding towards zero, and \code{RoundingStrategy.ROUND\_ONE} with zero digits after the decimal delimiter and also rounding towards zero.
\\

Rounding is implemented as strategy pattern, but an abstract class (\code{AbstractRoundingStrategy}) is introduced in the pattern (Figure \ref{quantity_overview}).
\code{AbstractRoundingStrategy} contains common methods such as \code{equals}, \code{hashcode} and getters, thus reducing code repetition.
\subsection{\code{Money} - A usecase for \code{Quantity}}
An object of the class \code{Money} is used to represent an amount of currency.
The following paragraphs detail the intended use, internal modelling and implementation of \code{Money}.
The UML model is given in Figure \ref{money_overview}.

\begin{figure}[ht]
	\centering
  \includegraphics[width=0.75\textwidth]{images/Money_Overview.eps}
	\label{money_overview}
	\caption{Money - Class Overview}
\end{figure}

A \code{Money} object can be instantiated by passing the numerical value as constructor parameter.
In this case, the metric \code{Metric.EURO} is used, as well as \code{RoundingStrategy.MONETARY} for the rounding strategy attribute.

For other currencies, a \code{Metric} parameter can be passed to the constructor along with a numerical paramter.
However, conversion between currencies is not supported, as it was not deemed necessary.

The rounding strategy cannot be overridden.
Internally, \code{Money} objects use four digits after the decimal delimiter for arithmetic operations to minimize the rounding error.
The \code{toString()} method, however, limits the output to the expected two digits after the decimal delimiter and appends the symbol of the associated \code{Metric}.

Two convenience instances exist: \code{Money.ZERO}, representing \euro{0,00}, and \code{Money.OVER9000}, representing an amount greater than \euro{9000,00}.

\subsection[]{\code{Unit} - Representing persons or other integral\protect\footnote{whole-number} items}
To represent integral items conveniently, the objects of class \code{Unit} can be used.
The rounding strategy is fixed for all instances to \code{RoundingStrategy.ROUND\_ONE} (Figure~\ref{quantity_overview}) and \code{Metric.PIECES} (Figure~\ref{money_overview}) is used as metric.
Convenience instances for amounts of zero, one and ten unit(s) exist (\code{Unit.ZERO}, \code{Unit.ONE}, and \code{Unit.TEN}; see Figure~\ref{quantity_overview}).
