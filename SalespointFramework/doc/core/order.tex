\newpage
\section{Order}
Orders are a new feature in \salespoint. The intention was to prepare the old databaskets for development of web applications and extend them to collaborate with our new inventory, providing the opportunity of individual pricing and final payment.

\begin{figure}[ht]
	\centering
  \includegraphics[width=1.0\textwidth]{images/Order_Overview.eps}
	\label{order_overview}
	\caption{Order - Class Overview}
\end{figure}

\subsection{\code{Order} - The main data structure}
We created the \code{Order} as main data structure to access the functionality above. An \code{Order} can be imagined as sheet of paper which basically consists of lines representing the ordered products (\code{OrderLine}) and lines that are not bound to a concrete product but which cause charge (\code{ChargeLine}). \code{ChargeLines} make it possible to add individual charge to orders. For example they can be used to define forwarding charges or other special charges produced by this order.
To identify the actor of this order the \code{UserIdentifier} from orderer is also be needed. The \code{SalesChannel} should be used to save information from which channel the order was created (e.g. telephone, internet or mail).

\code{Orders} are lifecycle-objects. The lifecycle covers four states which are defined by enumeration type \code{OrderEntryStatus}. After all the lifecycle has restrictions in changing states, which are transposed automatic by the according methods. \code{COMPLETED} is one of the final states and it's not possible to change the state of such \code{Orders}. The state machine below shows an example of a frequently practised lifecycle.

\begin{figure}[ht]
	\centering
  \includegraphics[width=1.0\textwidth]{images/Order_StateMachine.eps}
	\label{order_statemachine}
	\caption{Order - Lifecycle}
\end{figure}  

As you can see an OrderEntry can only completely modified in state \code{OPEN}. \code{PAYED}, \code{CANCELLED} and \code{COMPLETED} \code{Orders} are immutable!  Calling the \code{payOrder()} method changes the state and calls the accountancy to create a \code{PaymentEntry}. Ordered objects will be removed from inventory not until the \code{completeOrder()} method is called.  

\subsection{\code{OrderLine} - Representing ordered objects}
An \code{OrderLine} contains all information needed to identify product instances and deal with them. Every \code{OrderLine} represents one Product that contains all information about \code{ProductType} and \code{ProductFeatures}. Hence an \code{Order} contains an \code{OrderLine} for every ordered product. As well as \code{Orders}, \code{OrderLines} can also contain \code{ChargeLines}.

\subsection{\code{ChargeLine} - Representing additional charge}
A \code{ChargeLine} represents an additional charge for an \code{OrderLine} over and above the \code{OrderLine} value or an extra charge added to an \code{OrderEntry}.

\subsection{\code{OrderManager} - The interface to JPA}
To simplify the management of \code{OrderEntries} we have implemented the \code{PersistentOrderManager}. With \code{PersistentOrderManager} you can persist, update, find and remove \code{OrderEntries}. Contained objects like \code{OrderLines} and \code{ChargeLines} will also be persisted, updated or removed with there \code{OrderEntry} (Cascade).
\code{PersistentOrderManager} is a non-Entity class and cannot be persisted!

